%%% SETUP %%%
\documentclass{article}
\usepackage[utf8]{inputenc}
\usepackage{titling}
\usepackage{setspace}
\usepackage{hyperref}
\setcounter{secnumdepth}{0} %Table of contents without numbers

\newcommand{\subtitle}[1]{%
  \posttitle{%
    \par\end{center}
    \begin{center}\large#1\end{center}
    \vskip0.5em}%
}
\hypersetup{
    colorlinks=true,
    linkcolor=black,
    urlcolor=blue,
}

%%%%% /SETUP %%%%%
%%%%% Title page %%%%%
\title{Proof of Concept - Elaboration}
\subtitle{FOAR705 2019}
\author{Matthew Clark\\43695841}
\date{\vspace{-5ex}} %Used to remove date when using \maketitle%
\begin{document}
\doublespacing
\maketitle
%%%%% /Title Page %%%%%
%%%%% Table of Contents %%%%%
\newpage
\tableofcontents
%%%%% /Table of Contents %%%%%
%%%%% Entry 1 %%%%%
\newpage
\section{Elaboration I}
\subsection{Scope}
Before I begin in discussing the various technologies that could be used in my project, I have decided to narrow the scope for this research project such that I can focus more intently on it. Hence, my project will be focusing purely on using an API to gather a data set. More specifically, my scope includes:
\begin{enumerate}
    \item Identifying the scope of the data I want to collect. Because this is dependent on the specific API I use, this will be identified as I gain access to a specific API.
    \item Identifying what API I want to use. At this stage, I will be looking at the Beatport API.
    \item Developing a script in Python in order to access the API and gather the data based on Beatport's API's documentation
    \item Convert the retrieved JSON data into a spreadsheet
    \item Copy the spreadsheet and clean up the data
\end{enumerate}
\subsection{Technologies}
Beatport api; python in pycharm; excel; json; etc.
\subsection{Alternative Routes}
Other api's; scraping from web (charts, online magazines)



\end{document}
